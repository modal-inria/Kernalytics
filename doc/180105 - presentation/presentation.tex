\documentclass{beamer}

\mode<presentation> {
	\usetheme{CambridgeUS}      % or try Darmstadt, Madrid, Warsaw, ...
	\usecolortheme{default} % or try albatross, beaver, crane, ...
	\usefonttheme{default}  % or try serif, structurebold, ...
	\setbeamertemplate{navigation symbols}{}
	\setbeamertemplate{caption}[numbered]
}

\usepackage[utf8]{inputenc}
\usepackage[T1]{fontenc}
\usepackage[french]{babel}

\usepackage{amsmath}
\usepackage{amssymb}
\usepackage{alltt}
\usepackage{graphicx}
\usepackage{subcaption}
\usepackage{tabularx}

\usepackage{makecell}

\usepackage[fixlanguage]{babelbib}
\selectbiblanguage{french}

\title[Kernalytics]{Kernalytics - Méthodes à noyaux et modélisation}
\author[VK]{Vincent KUBICKI - InriaTech}
\institute[Inria]{Inria Lille - Nord Europe}
% \date{17 Octobre 2017}

\begin{document}

\begin{frame}[plain]
	\titlepage
\end{frame}

\section{Introduction}

\begin{frame}{Kernalytics - Généralités}
	\begin{itemize}
		\item Contrat Coreye.
		\item Preuve de concept pour utiliser les noyaux.
		\item Pallier aux difficultés liées au C de KernSeg.
		\item Scala et programmation fonctionelle.
		\item Gestion des données multivariées.
	\end{itemize}
\end{frame}

\section{Mathématiques}

\begin{frame}{Segmentation à noyaux}
	\begin{itemize}
		\item Adaptation directe de "New Efficient Algorithms for Multiple Change-Point Detection with kernels", de Célisse et al.
		\item Implémentation utilisant la programmation dynamique.
		\item Sélection du nombre optimal de segments via une heuristique de pente.
	\end{itemize}
\end{frame}

\begin{frame}{Modularité}
	\begin{columns}
		\begin{column}{0.3\textwidth}
			\begin{block}{Commun}
				\begin{itemize}
					\item Matrice de Gram
					\item KerEval
					\item ...
				\end{itemize}
			\end{block}
		\end{column}
		\begin{column}{0.3\textwidth}
			\begin{block}{Noyaux}
				\begin{itemize}
					\item Linéaire
					\item Polynomial
					\item Gaussien
					\item Laplacien
				\end{itemize}
			\end{block}
		\end{column}
		\begin{column}{0.3\textwidth}
			\begin{block}{Algorithmes}
				\begin{itemize}
					\item Régression
					\item K-Means
					\item Test d'égalité de distributions
					\item Segmentation
				\end{itemize}
			\end{block}
		\end{column}
	\end{columns}
\end{frame}

\begin{frame}{Structures algébriques}
	\begin{itemize}
		\item Tous les noyaux sont génériques, nécessitant un type de donnée et une structure algébrique.
		\item Exemple: la segmentation de matrice nécessite seulement un produit interne dans l'espace des matrices réelles (Frobenius par exemple).
		\item Déduction des structures algébriques (i.e. produit interne $\rightarrow$ norme $\rightarrow$ distance).
		\item Hiérarchie algébrique validée par typage.
	\end{itemize}
\end{frame}

\begin{frame}{Programmation fonctionnelle}
	\begin{columns}
		\begin{column}{0.45\textwidth}
			\begin{block}{Immutabilité}
				\begin{itemize}
					\item Aucune variable.
					\item Aucune boucle.
					\item Généralisation des lapply, reduce, etc... de R
				\end{itemize}
			\end{block}
		\end{column}
		\begin{column}{0.45\textwidth}
			\begin{block}{Composition de fonctions}
				\begin{itemize}
					\item Combinaison linéaire de noyaux.
					\item Code est concis et de haut niveau.
					\item Simplicité d'intégration d'un nouveau noyau, sous la forme d'une fonction $(X, X) \rightarrow \mathbb{R}$
					\item Simplicité pour ajouter un nouveau type de données, sous la forme d'une fonction $\text{Chaîne de caractères} \rightarrow X$.
				\end{itemize}
			\end{block}
		\end{column}
	\end{columns}
\end{frame}

\section{Informatique}

\begin{frame}{Génie logiciel}
	\begin{itemize}
		\item Utilisation d'une d'algèbre linéaire / statistiques: Breeze.
		\item Code est compilé: beaucoup de vérifications sont effectuées lors de la compilation, et l'exécution est rapide.
		\item Tests unitaires très faciles à implémenter, gestion des exceptions limpide.
		\item Code compilé pour la JVM, facilité de déploiement multiplateforme.
		\item Maintenance plus simple que du C++ (i.e. système de build trivial).
	\end{itemize}
\end{frame}

\begin{frame}{Utilisation}
	\begin{block}{Format des données}
		\begin{itemize}
			\item 2 fichiers d'entrée en csv: données et descripteurs
			\item 1 colonne par variable en données, 1 colonne par noyau en descripteurs
			\item Plusieurs noyaux possibles pour une même variable, par exemple
		\end{itemize}
	\end{block}
	\begin{columns}
		\begin{column}{0.45\textwidth}
			\begin{block}{Données}
				\begin{itemize}
					\item Nom de variable
					\item Type de donnée
					\item Une observation par ligne
					\item ...
				\end{itemize}
			\end{block}
		\end{column}
		\begin{column}{0.45\textwidth}
			\begin{block}{Descripteurs}
				\begin{itemize}
					\item Nom de variable
					\item Poids
					\item Noyau et paramètres
				\end{itemize}
			\end{block}
		\end{column}
	\end{columns}
\end{frame}

\section{Conclusion}

\begin{frame}{A faire}
	\begin{itemize}
		\item Créer un paquet R en utilisant rscala (wrapper léger, prototype déjà testé).
		\item Intégrer des noyaux plus avancés.
	\end{itemize}
\end{frame}

\end{document}